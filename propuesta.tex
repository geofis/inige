\documentclass[12pt,letterpaper,spanish]{article}
\usepackage[utf8]{inputenc}
\usepackage[spanish]{babel}
\usepackage[right]{lineno}
\usepackage{cite}
\usepackage{amsmath}
\usepackage{nameref}
\usepackage[hidelinks]{hyperref}
\usepackage[left=4cm,right=3cm,top=4cm,bottom=3cm]{geometry}
\usepackage[all]{nowidow}
%\vfuzz=1pt
\usepackage[hang]{footmisc}
\setlength\footnotemargin{10pt}
\usepackage{enumitem}
\setitemize{noitemsep,topsep=0pt,parsep=3pt,partopsep=3pt}
\usepackage{lastpage}
\newcommand\numpag{\pageref{LastPage}}

\title{Instituto de Investigaciones Geográficas \\ INIGE-UASD \\ \large Propuesta de creación}
\author{Martínez Batlle, José Ramón\footnote{Correo electrónico: joseramon@geografiafisica.org\newline Versión 0.1, agosto de 2018}}
\newcommand{\ent}{Instituto de Investigaciones Geográficas}
\newcommand{\entacron}{INIGE}


\begin{document}

\makeatletter
\begin{center}
{\LARGE \@title \par}
\makeatother
Martínez Batlle, José Ramón\footnote{Correo electrónico: joseramon@geografiafisica.org\newline Versión 0.1, agosto de 2018\newline Ofrecí este documento al Decano Radhamés Silverio y  a Máximo Portorreal, pero lo escribí sobre todo porque lo que considero necesario es crear instituto nuevo, no una reforma de nada anterior.}
\end{center}

\tableofcontents%Tabla de contenidos
\addcontentsline{toc}{section}{Resumen}

\linenumbers

\section*{Resumen} \label{resumen}

\section{Introducción} \label{introduccion}

La escasa longitud de este documento (\numpag{} páginas) facilita su lectura, así como su actualización al momento de cumplirse la vigencia estipulada.

En mi entorno académico más directo percibo (ay la percepción) que la investigación se entiende como una acción compleja que se realiza con tres elementos: sofisticados equipos, mucho dinero y personas que parecen iluminadas. Lamentablemente, tarde me di cuenta de que, si bien se requiere de elementos especiales que no son comúnmente encontrados en nuestro diario vivir, investigar con poco es posible. Igualmente, se pueden publicar resultados que puedan contrastarlos otras personas que, a fin de cuentas, tampoco son iluminadas.

Por otra parte, estoy convencido de que es legítimo investigar con miras a publicar en revistas de lujo, pero para ello se requiere un volumen considerable de los tres elementos citados anteriormente. Dicho a secas, si se quiere publicar en revistas de lujo, se tiene que estar en disposición de pagar un alto costo. Alternativamente, aspirar a hacer ciencia con poco parece un buen comienzo para un \ent{} (\entacron) recién desempacado.

No veo factible planificar para más de un año. En su lugar, propongo una planificación adaptativa, sugiriendo acciones que produzcan resultados tangibles en el corto plazo, pero que a su vez contribuyan con la misión global. Propongo que el \entacron{} adopte la misión de \textbf{aportar nuevo conocimiento en geografía}, y que esta propuesta tenga una \textbf{vigencia de un año a partir del momento de su creación}.

Proponer una estructura de investigación nueva capaz de aportar conocimientos con pocos recursos constituye un gran desafío. Este es el objetivo principal del presente documento.

\section{Resultados esperados} \label{cap:resesp}

Crear una nueva entidad requiere de un esfuerzo creativo y de revisión periódica. Aplicar la creatividad en estructuras preexistentes resulta más complicado\footnote{En junio de 2014 formulé una propuesta de reforma del Instituto Geográfico Universitario (IGU)\cite{martinez2014futuroigu}. Todas las negativas que recibí me dejaron una cuestión clara: el deseo de cambiar las cosas, la libertad y la creatividad no eran virtudes comunes entre las personas a las que les presenté el documento, muchas de las cuales hasta tuve que enfrentarlas. Esta nota al pie es lo único que pretendo dedicar al IGU, así como a todas aquellas personas que en su momento prefirieron ``no mover ficha'' y continuar en su zona de confort.}. Pero más complicado aún resulta proponerse alcanzar resultados de investigación en corto plazo. Esta sección se ocupa de dicha tarea. Relaciono a continuación los resultados que propongo alcanzar en el primer año:

\begin{itemize}
\item $N~investigadorxs \times 2$ publicaciones (e.g. $5 \times 2=10$) científicas producidas. Como mínimo, toda investigación debería producir manuscritos que se envíen a revistas con revisión por pares (\textit{peer review}) o a servidores de ``prepublicaciones'' (\textit{preprint servers}). Los reportes periódicos serán ``plepla'' si no se publican artículos científicos o se transforman en documentos \texttt{arXiv} \cite{revisionpares, preprints, prepublicacion, arxiv}.
\item $N~investigadorxs \times 2$ presentaciones realizadas en congresos, jornadas o eventos organizados \textit{ex profeso}. Sólo se presentarán estudios remitidos a revistas o \textit{preprint servers}.
\begin{sloppypar}
\item $N~investigaciones~en~marcha$ documentos de preproducibilidad/reproducibilidad con los resultados alcanzados y los métodos aplicados (Markdown es una alternativa idónea para este tipo de documentos\cite{wiki2018markdown}). Asumo preproducir en el sentido expresado en\cite{stark2018before}, donde se sugiere la redacción de documentos que permitan a cualquier persona reproducir los resultados obtenidos.
\end{sloppypar}
\item Aquí lo dejo. Para quien no investiga, 2 investigaciones por año podría parecer poco. Para quien investiga, esta cifra podría parecer grande. Por tal razón, la o las preguntas de investigación a responder deberán formularse teniendo en cuenta estos estrechos plazos y las condiciones logísticas basadas en escasos recursos. Lo mismo aplica para los sitios de muestreo, mejor cerca que lejos.
\end{itemize}

\section{Materiales, métodos, personal} \label{cap:matmetpers}

\subsection{Materiales y servicios} \label{cap:matser}

En principio, con poco equipamiento debería poderse aportar nuevo conocimiento a la humanidad. Los materiales necesarios están estrechamente relacionados con los resultados que el \entacron{} se plantee alcanzar (ver sección ``\nameref{cap:resesp}''). Por lo tanto, en un primer año de investigación, el \entacron{} requeriría del siguiente equipamiento:

\begin{itemize}
\item Un local ``digno'' (ja ja, qué iluso).
\item Energía eléctrica ¿permanente? (ja ja, otra vez).
\item Inversor, con sus baterías, por supuesto (ya, sí, claro).
\item Contrato de conexión a Internet, preferiblemente de 10 o 100 Mbps.
\item Línea telefónica.
\item Módem portátil con conexión por contrato.
\item Servicio de copia de seguridad en la nube
\item Al menos 5 discos duros para copias locales, algo que parece pertenecer al pasado, pero no en RD.
\item $N~investigadorxs \times 1$ computadoras sin sistema operativo ni otro software preinstalado.
\item Servidor sin software preinstalado.
\item Material a considerar a futuro: calibradores pie de rey, cintas de medir de 50\,o\,100\,m y de 3\,o\,5\,m, gravelómetro, juego de tamices, balanza de precisión, medidor multiparamétrico, lupa binocular con hasta 90X aumentos, 2+ placas de Petri sin divisiones, 4+ con divisiones, frascos.
\item Consumibles, insumos.
\end{itemize}

Cualquier otro material requerido debería justificarse debidamente. Realizar estudios que no impliquen la adquisición de compras debe ser la norma. Tampoco servirán las promesas de que ``la gestión será diferente esta vez'', porque lo que importan son los resultados\footnote{Me he visto a mí mismo y a más personal investigador buscando cotizaciones, yendo a aduanas, llamando a proveedores, solicitando tintas, estando pendientes de un viático el día anterior al viaje, y otras complicaciones. Si la realidad es que la UASD no tiene la capacidad de realizar estos procedimientos en un tiempo prudente, entonces la investigación tendrá que adaptarse a ello. Bahoruco siempre podrá verse en fotos, o quizá en imágenes de satélite.}. Por lo tanto, si una pregunta de investigación requiere de equipos costosos para ser respondida, o si ésta debe realizarse en lugares lejanos e inaccesibles, entonces el \entacron{} no es el lugar apropiado para responderla, al menos durante su primer año. Si el \entacron{} llega a producir sin dinero y con escasos materiales, entonces quizá esté preparado para investigar con grandes cantidades de recursos.

\subsection{Métodos}

La lista de métodos se adapta a la lentitud de los procedimientos administrativos en la UASD. Resumo aquí los principales métodos y consideraciones metodológicas que se deberían implementar:

\begin{itemize}
\item El ``lugar de estudio'' de cada investigación debería ser accesible en cualquier medio de transporte desde el \entacron{}. Estudiar Gazcue, la Zona Universitaria, el Jardín Botánico Nacional, o los parques Mirador del Norte y Mirador del Sur podría no sonar ``sexy'', pero es realizable. Las ``lejanas latitudes'' cuestan mucho dinero y dañan carros privados.
\item Es difícil generalizar, pero el tamaño deseable de la muestra en cada investigación debería ser de más de 2000 elementos y de 10 o más sitios, preferiblemente un mínimo de 30. Sí, hay que tomar muestras, pero alternativamente se pueden utilizar datos ya recogidos por otras iniciativas, o utilizar datos almacenados en colecciones, servidores o geoservidores\footnote{Basta recordar que los satélites de la empresa Planet\cite{planet2018planetexplorer} tienen una tiempo de revisita de 3 días (ofrecen cuenta para investigación), Landsat\cite{nasa2013land} ``pasa'' cada 16, Sentinel\cite{scihub2018copernicus} cada 12. El Instituto Smithsonianiano tiene bases de datos gigantescas (como la de BCI) a disposición de cualquier usuario que se registre. Existen bases de datos espaciales como la de nombres geográficos\cite{geonames2018ngia}, o la de cobertura arbórea mundial\cite{hansen2015global}. La ristra de servicios de información disponible es muy larga, pero todos tienen en común que sólo exigen honrar los créditos, y todos promueven que se produzcan nuevos conocimientos con sus fuentes.}.
\item Métodos basados en la estadística y el análisis espacial. Las excepciones deberán justificarse debidamente, e igualmente la revista donde se publicará un estudio cualitativo deberá seleccionarse con antelación.
\item La recogida de datos se deberá lograr con software de código abierto y preferiblemente usando los celulares de los propios investigadores. Ya intenté recoger datos en el campo con costosos equipos, pero los celulares los superan por mucho. ¡Al fin una utilidad para los celulares!
\item El análisis de datos igualmente deberá lograrse con software de código abierto y, siempre que sea posible, mediante el intérprete de línea de órdenes (CLI).
\item Analizar datos mediante interfaz gráfica sólo se recomienda cuando no sea posible mediante el CLI (algo poco habitual).
\item Se ``perdona'' la visualización y edición manual de datos espaciales, por tratarse de tareas en las que la interfaz gráfica es mucho más eficiente.
\item La adquisición de software privativo sólo se justifica si el análisis a realizar no está implementado en alternativas de código abierto (algo poco habitual).
\end{itemize}


\subsection{Personal} \label{cap:personal}

\begin{itemize}
\item $N~investigadorxs$, cifra óptima entre 3 y 5 para el primer año, aunque este extremo dependerá de muchos factores, como la disponibilidad de personal, los honorarios que la UASD esté en disposición de pagar, las condiciones de trabajo que se ofrezcan, entre otros.
\item Este personal debe pertenecer al área docente. La experiencia me ha demostrado que la estructura basada en personal administrativo (denominados ``empleados'' en la UASD, aunque empleados somos todos) NO es factible para la investigación. Hay muchos argumentos sobre esta idea, pero me da una pereza enorme ponerlos aquí.
\item La contratación externa es una posibilidad que debería abrirse en este caso.
\item Una cualidad a preferir entre el personal investigador a contratar sería su motivación propia para la investigación. Esto se puede medir con relativa facilidad: si tras llegar el invierno nuclear, o previo aviso, se decide abandonar la investigación, entonces no hay la suficiente motivación.
\item Un (1, la unidad, singular) secretario/a. La experiencia me demuestra que éste es el único personal administrativo necesario.
\item Sí, quien haya llegado a este punto se debe estar riendo o cuestionando al iluso redactor. Me pidieron una propuesta y la hice sin techo de cristal.
\end{itemize}


\section{Consideraciones finales} \label{cap:considfinales}

¡APROBADO!

\bibliographystyle{plos2015}
\bibliography{biblio}


\end{document}
